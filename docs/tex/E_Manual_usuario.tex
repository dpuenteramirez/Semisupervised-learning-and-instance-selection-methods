\apendice{Documentación de usuario}

\section{Introducción}
En esta sección se detallan los requerimientos de la aplicación, su instalación y despliegue (en el caso de \texttt{UBUMLaaS}) y se acompañan de una serie de indicaciones y consejos para su correcto uso.

De igual manera que en el Manual del Programador cada parte del proyecto, \texttt{IS-SSL} y \texttt{UBUMLaaS}, se describirá por su propio lado, de tal manera que aunque haya aspectos comunes, cada una su propia documentación de usuario.

\section{UBUMLaaS}
\subsection{Requisitos de usuarios}
Los requisitos mínimos para poder hacer uso de \texttt{UBUMLaaS} son:
\begin{itemize}
\item Disponer de una conexión a Internet.
\item Hacer uso de navegador web con soporte a HTML5.
\item Tener habilitado JavaScript en el navegador.
\item Tener una cuenta en la plataforma.
\end{itemize}
\subsection{Instalación}
Al tratarse de un producto web no se requiere de ningún tipo de instalación. Los navegadores Google Chrome, Mozilla Firefox, Safari y Microsfot Edge son soportados\footnote{Todos ellos han sido probados por el equipo de desarrollo y usuarios encuestados a  los que se les proporcionó un documento de uso básico y lo usaron en sus dispositivos cotidianos.}, siempre y cuando se encuentren en versiones compatibles con HTML5 y tengan activado el uso de JavaScript.

Independientemente del dispositivo de uso (ordenador de sobremesa, portátil, tableta o móvil), se requiere de conexión a Internet como es lógico. Pero no necesita de permisos adicionales, ha sido desarrollada de tal manera que utiliza la sesión local del navegador sin necesidad del uso de \textit{cookies}.

Aunque se hizo un intento de traducción a los lenguajes más comunes, finalmente se encuentra en inglés de forma única.

\subsection{Manual del usuario}



%%%%%%%%%%%%%%%%%%%%%%%%%%%%%%%%%%%%%%%%%%%%%%%%%%%%%%%%%%%%%%%%%%%%%%
\section{IS-SSL}
\subsection{Requisitos de usuarios}
Los requisitos mínimos para poder hacer uso de \texttt{IS-SSL} son:
\begin{itemize}
\item Tener instalado Python 3.7+.
\item Tener instalado y configurado \texttt{PIP} o \texttt{Conda}.
\item Disponer de un editor de textos.
\item Tener instaladas las bibliotecas necesarias para su correcto funcionamiento.
\end{itemize}

\subsection{Instalación}

Por comodidad para el usuario, \texttt{IS-SSL} se ha dividido en dos bibliotecas, una formada por los algoritmos de selección de instancias, y una segunda por aquellos algoritmos de aprendizaje semi-supervisado.

El proceso de instalación de cualquiera de las dos bibliotecas es muy sencillo, siendo integrable en cualquier fichero de requerimientos, ya sea para \texttt{PIP} o \texttt{Conda}.

Las dos bibliotecas se encuentran publicadas en PyPI\footnote{\textit{Python Package Index} es un repositorio de \textit{software} para el lenguaje de programación de Python.} desde su versión 1.0, la cual fue una primea versión alpha estable con los primeros algoritmos publicados. 
La versión 3.0 es la versión estable (la final) que se ha publicado.

\imagenFlotante{../img/anexos/manual-usuario/PyPI-IS}{Vista de la biblioteca de algoritmos de selección de instancias en PyPI.}{PyPI-IS}
\imagenFlotante{../img/anexos/manual-usuario/PyPI-SSL}{Vista de la biblioteca de algoritmos de aprendizaje semi-supervisado en PyPI.}{PyPI-SSL}

Para realizar la instalación se deben seguir los siguientes pasos para cualquier LIB, LIB $\in \lbrace$ IS-DNX, SSL-DNX$\rbrace$.

\begin{enumerate}
\item Acceder a PyPi, desde~\cite{PyPI}.
\item Introducir en el campo de búsqueda <<LIB>>.
\item Seleccionar la biblioteca correspondiente de entre la lista mostrada.
\item Copiar el comando de instalación.
\item Abrir una terminal con soporte a Python y \texttt{PIP}.
\item Introducir el comando copiado.
\item En caso de que se nos pregunte si se quiere proceder con la descarga, indicar que sí con una S en caso de que esté en español, o con Y en el caso inglés/internacional.
\item Cuando finaliza la instalación, la biblioteca se encontrará lista para su uso.
\end{enumerate}

\imagenFlotante{../img/anexos/manual-usuario/PIP-IS}{Instalación de la biblioteca de selección de instancias.}{PIP-IS}
\imagenFlotante{../img/anexos/manual-usuario/PIP-SSL}{Instalación de la biblioteca de semi-supervisado.}{PIP-SSL}


\subsection{Manual del usuario}


