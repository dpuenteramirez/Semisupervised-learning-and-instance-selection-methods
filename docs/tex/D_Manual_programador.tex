\apendice{Documentación técnica de programación}

\section{Introducción}
En este anexo se va a describir con detalle la documentación técnica de programación. Se describirá la estructura de directorios que posee, la instalación del propio entorno de desarrollo, cómo llevar a cabo su compilación, instalación y ejecución; además de las pruebas que se han realizado.

Se debe recordar que el proyecto se encuentra dividido en dos repositorios diferenciados, UBUMLaaS e IS-SSL~\footnote{Biblioteca de algoritmos de selección de instancias y aprendizaje semi-supervisado programado.}; es por ello que, se dividirá en dos secciones respectivamente, y tantas subsecciones como son necesarias para cada uno de ellos.

\section{UBUMLaaS}

\subsection{Estructura de directorios}
La estructura del repositorio es la siguiente:
\begin{itemize}
\tightlist
\item \texttt{/}: raíz del proyecto, aquí se encuentra el README, la licencia, los ficheros de configuración de las pruebas de integración y despliegue continuo (CI-CD), junto con los ficheros de requisitos para \texttt{conda} y \texttt{pyenv}. 
\item \texttt{/lib/}: librerías utilizadas por el sistema.
\item \texttt{/lib/is\_ssl}: librería propia de métodos de selección de instancias y aprendizaje semi-supervisado.
\item \texttt{/lib/scikit\_ml\_learn\_data/meka/meka-release-1.9.2/}: librería \texttt{Meka} en su versión 1.9.2.
\item \texttt{/lib/skmultilearn/}: librería \texttt{scikit-multilearn}.
\item \texttt{/lib/unofficial\_weka\_packages/}: algoritmos de ADMIRABLE.
\item \texttt{/lib/wekafiles/}: algoritmos concretos de \texttt{weka}.
\item \texttt{/test/*}: ficheros de prueba CI-CD.
\item \texttt{/ubumlaas/}: directorio principal de la plataforma.
\item \texttt{/ubumlaas/admin/}: contiene toda la parte de \textit{backend} de administración.
\item \texttt{/ubumlaas/core/}: contiene el \textit{backend} de las vistas de índice y acerca de.
\item \texttt{/ubumlaas/default\_datasets/}: conjuntos de datos por defecto que se añaden a los nuevos usuarios.
\item \texttt{/ubumlaas/error\_pages/}: contiene el \textit{backend} de las vistas de error.
\item \texttt{/ubumlaas/experiments/}: contiene el \textit{backend} para la realización de experimentos.
\item \texttt{/ubumlaas/experiments/algorithm/}: contiene las métricas para el análisis del modelo entrenado.
\item \texttt{/ubumlaas/experiments/execute\_algorithm/}: contiene opciones de ejecución para cada librería.
\item \texttt{/ubumlaas/experiments/views/}: control de las vistas relacionadas con los experimentos.
\item \texttt{/ubumlaas/jobs/}: descripción de \textit{RQ Worker Builder}.
\item \texttt{/ubumlaas/static/}: contiene los ficheros estáticos de la plataforma.
\item \texttt{/ubumlaas/static/avatars/}: contiene las imágenes de perfil de cada usuario.
\item \texttt{/ubumlaas/static/css/}: contiene el código \texttt{CSS} del \textit{frontend}.
\item \texttt{/ubumlaas/static/img/}: contiene las imágenes que aparecen en la pltaforma.
\item \texttt{/ubumlaas/static/js/}: contiene el código \texttt{JavaScript} del \textit{frontend}.
\item \texttt{/ubumlaas/templates/}: ficheros \texttt{HTML}.
\item \texttt{/ubumlaas/templates/admin/}: ficheros web de administración.
\item \texttt{/ubumlaas/templates/blocks/}: ficheros web de bloques que se añaden sobre otros documentos web.
\item \texttt{/ubumlaas/templates/error\_pages/}: ficheros web de errores (403, 404, \dots)
\item \texttt{/ubumlaas/templates/modals/}: ficheros para la representación de modales.
\item \texttt{/ubumlaas/users/}: contiene el \textit{backend} de las actividades relacionadas con el usuario.
\item \texttt{/ubumlaas/weka/}: contiene los ficheros de configuración de \texttt{Weka} y su \texttt{VM}.

\end{itemize}

\subsection{Manual del programador}

\subsection{Compilación, instalación y ejecución del proyecto}

\subsection{Pruebas del sistema}

\section{IS-SSL}

\subsection{Estructura de directorios}
La estructura del repositorio es la siguiente:
\begin{itemize}
\tightlist
\item \texttt{/}: raíz del proyecto, aquí se encuentra el README, la licencia, los ficheros de configuración de PIP, los ficheros de configuración de las pruebas de integración y despliegue continuo (CI-CD); y, el fichero de requisitos.
\item \texttt{/datasets/*}: conjuntos de datasets en formatos \texttt{csv} y \texttt{arff}, normalizados y no normalizados.
\item \texttt{/docs/}: documentación del proyecto.
\item \texttt{/docs/img/}: imágenes utilizadas en la documentación.
\item \texttt{/docs/img/anexos/*}: imágenes utilizadas en los anexos.
\item \texttt{/docs/img/draws/}: diagramas en su formato original.
\item \texttt{/docs/img/memoria/*}: imágenes utilizadas en la memoria.
\item \texttt{/hypothesis/*}: primera aproximación a la investigación realizada.
\item \texttt{/implementation\_tests/}: conjunto de pruebas de validación sobre los algoritmos implementados.
\item \texttt{/instance\_selection/}: algoritmos implementados de selección de instancias.
\item \texttt{/instance\_selection/utils/}: métodos de apoyo comunes a los algoritmos de selección de instancias.
\item \texttt{/misc/}: contiene archivos varios de formato para el repositorio (cabeceras, logos, etc.).
\item \texttt{/semisupervised/}: algoritmos implementados de aprendizaje semi-supervisado.
\item \texttt{/semisupervised/utils/}: métodos de apoyo comunes a los algoritmos de aprendizaje semi-supervisado.
\item \texttt{/utils/}: diferentes clases y métodos de apoyo comunes tanto a selección de instancias como a semi-supervisado.
\end{itemize}

\subsection{Manual del programador}

\subsection{Compilación, instalación y ejecución del proyecto}

\subsection{Pruebas del sistema}
