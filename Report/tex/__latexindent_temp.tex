\capitulo{3}{Conceptos teóricos}

El proyecto tiene una relación directa con la minería de datos y los conceptos que lo rodean. 

\section{Minería de datos}

Según IBM \cite{IBM-WhatisDataMining}, podemos definir la minería de datos, o descubrimiento de conocimiento
en los datos \textit{knowledge Discovery in Database}, como el proceso de descubrir patrones y otra
información a partir de grandes conjuntos de datos. 

Las técnicas de minería de datos principales se pueden dividir en función de sus propósitos principales.
\begin{enumerate}
    \item Descripción del conjunto de datos objetivo.
    \item Predicción de resultados mediante el uso de algoritmos de aprendizaje automático.
\end{enumerate}


\subsection{Proceso de minería de datos}
El proceso seguido en la minería de datos es muy directo. Comienza con la recogida de los datos que van a ser
tratados. y finaliza con la visualización de la información extraída de éstos. 
Los científicos de datos describen los datos a través de sus observaciones de patrones, asociaciones y correlaciones. A su vez se pueden clasificar y agrupar los datos utilizando métodos de clasificación y regresión.

Uno de los marcos de referencia más importantes en el proceso de minado de datos es CRISP-DM, \textit{Cross Industry Standard Process for Data Mining}. Desarrollado por un consorcio de empresas involucradas en la minería de datos. \cite{Chapman2000CRISPDM1S}

\imagen{../img/CRISP-DM}{Enfoque CRISP de la minería de datos.}


En \cite{KOTU201517} se divide el proceso de la minería de datos en 5 etapas o pasos principales: establecimiento de los objetivos y comprensión del problema, recopilación y preparación de los datos, desarrollo del modelo, aplicación del modelo y la evaluación de los resultados y despliegue en producción.

\begin{enumerate}
   \item \textbf{Establecer los objetivos y comprensión del problema.}
    La primera etapa puede resultar la más complicada del proceso. Todas las partes interesadas deben de estar presentes y de acuerdo en la definición del problema que se va tratar, esto incluye tanto a los científicos de datos como las terceras partes involucradas o interesadas. 
    Este procedimiento ayuda a la formulación de las preguntas de los datos y los parámetros a utilizar en el proyecto. Si se trata de un proyecto empresarial, se debe hacer un estudio o investigación adicional para comprender el contexto de la empresa.
    \item \textbf{Preparación de los datos.}
    Con el alcance del problema definido ya se puede comenzar a identificar qué conjunto de datos será el más efectivo o representativo con el fin de comenzar a dar respuesta a las preguntas formuladas en el proceso anterior.
    
    Una vez se tienen todos los datos recogidos comienza el proceso de pre-procesado de los mismos. Este proceso se basa en la limpieza de los datos con el fin de eliminar cualquier posible ruido, entendiéndose por ruido los datos duplicados, los valores perdidos y aquellos atípicos; aquellos que puedan causar problemas a la resolución del problema o generen incertidumbre.
    En determinados conjuntos de datos se puede hacer una reducción de dimensiones. Consiste en la reducción del número de dimensiones que poseen las instancias recogidas, con el fin de eliminar aquellas que no sean realmente representativas o significativas, este proceso reduce la complejidad de los cálculos posteriores. Por contrapartida hay que conocer cuáles serán los predictores con mayor relevancia en el problema para garantizar una precisión "óptima" del modelo.
    \item \textbf{Desarrollo del modelo.}
    Según \cite{KOTU201517} el modelo es la representación abstracta de los datos y sus relaciones en un conjunto de datos concreto. Actualmente existen cientos de algoritmos que se pueden utilizar, habitualmente proceden de campos como la ciencia de datos, \textit{machine learning}, o la estadística.
    Se debe tener el conocimiento suficiente para entender como funciona el algoritmo para poder configurar correctamente los parámetros que este va a utilizar en base a los datos y el problema de negocio que estamos resolviendo. 
    
    Los modelos en función de como resuelvan el problema que se les presenta se pueden clasificar en:
    \begin{enumerate}
        \item Regresión.
        \item Análisis de asociación.
        \item \textit{Clustering.}
        \item Detección de anomalías.
    \end{enumerate}
    
    % TODO: seguir con esto https://books.google.es/books?hl=es&lr=&id=dRHoAwAAQBAJ&oi=fnd&pg=PP1&dq=Chapter+2+-+Data+Mining+Process+VijayKotuBalaDeshpandePhD&ots=SCVjIlON9w&sig=aYkEFyM_zrlNEVq7pjx5BpBYiUY#v=onepage&q&f=false
    
    \item \textbf{Evaluación y visualización de los resultados.}
\end{enumerate}